\chapter*{General Introduction}
Robotics has emerged as a pivotal technology in fields ranging from industrial automation and research to education and personal innovation. Despite rapid technological advances, a gap persists in providing accessible, versatile, and modular platforms that can cater to a diverse audience, ranging from researchers and students to hobbyists and ordinary users with minimal technical background. This thesis addresses this gap by introducing a set of modular robotic platforms; Bulky, Thegill, and µtomba, Dronegaze, alongside controllers like iLite and software tools for development and tinkering and interfacing which are designed with adaptability and ease of use in mind.

At the heart of this work is the idea of modularity as a means to bridge diverse domains from rigorous academic research and hands-on educational experimentation to hobbyist tinkering and everyday problem solving. Bulky is designed as a precision rover that integrates a flexible sensor-ready framework, exemplifying how a platform can evolve by accommodating custom modules. Thegill challenges conventional boundaries by operating seamlessly across terrestrial and aquatic environments, pushing the envelope on how modular components can harmonize mechanical adaptability with sophisticated control strategies. Meanwhile, µtomba distills robotics to its essentials, replacing rigid circuit boards with a breadboard setup to foster an environment of learning, rapid prototyping, and creative exploration.

Together, these platforms are not merely products with predefined applications; each platform embodies a distinct set of features and design philosophies while adhering to a common vision: to democratize robotics by offering modular and upgradable solutions that can be customized, extended, and integrated into various applications. They are conceptual testbeds that encourage users to tap on robotics via open design and abstraction. This thesis systematically explores the theoretical foundations of modular design, the practical integration of mechanical, software and electronic systems, and the experimental validation of these concepts. By doing so, it aims to demonstrate that embracing modularity and abstraction can transform robotics into a universally accessible and continually evolving field, empowering researchers, students, enthusiasts, and even everyday users to craft adaptive solutions in parallel of what is being developed in the industrial world.

This thesis documents the philosophy, problematic, design, implementation, and evaluation of these platforms, demonstrating how a modular approach can address a variety of challenges in robotics. It presents the theoretical foundations of modular design, details the system architecture, and evaluates the platforms through experimental testing. Ultimately, the work illustrates how such adaptable solutions can serve as universal products, empowering users to explore innovative applications in research, education, and beyond.

\section{Introduction to Modularity in Robotic Systems}
Modularity is a design philosophy that emphasizes constructing complex systems from smaller, self-contained components—commonly referred to as modules\footnote{Modules are discrete units of functionality that can be independently developed, tested, replaced, or upgraded.}. In the context of robotic platforms, this approach involves decomposing the system into discrete parts, each encapsulating specific functionalities such as sensing, actuation, communication, or processing, thereby simplifying design and troubleshooting.

At its core, modularity offers several significant benefits:

\begin{itemize}
  \item \textbf{Reusability}\footnote{Reusability refers to the ability to use developed modules in various configurations or projects, reducing overall development time and costs.}: Successfully developed and tested modules can be reused across different configurations or projects.
  \item \textbf{Flexibility}\footnote{Flexibility is the capacity to interchange or reconfigure modules to adapt to new requirements or experimental setups.}: Standardized interfaces\footnote{Interfaces are predefined methods for communication between modules, ensuring compatibility and ease of integration.} and communication protocols enable modules to be interchanged or reconfigured, a feature particularly valuable in rapid prototyping and iterative design.
  \item \textbf{Scalability}\footnote{Scalability describes the ease of integrating additional modules into an existing system as project requirements evolve.}: New modules can be integrated without necessitating a complete redesign, making it straightforward to expand functionality or incorporate emerging technologies.
  \item \textbf{Ease of Maintenance}\footnote{Ease of Maintenance refers to the capacity to replace or upgrade individual modules independently, thereby minimizing disruptions to the overall system.}: When a single module fails or becomes outdated, it can be replaced or upgraded without affecting the entire system.
\end{itemize}

In educational environments, modular robotic platforms serve as excellent teaching tools. They allow students to visualize and understand the intricate relationships between individual system components and the overall architecture. By experimenting with different module combinations, learners gain practical insights into system integration, problem-solving, and the inherent trade-offs in design decisions.

From a development perspective, modularity supports a more agile workflow. It enables parallel development, where different teams or individuals work simultaneously on separate modules. Upon integration, the overall system benefits from collective innovations, ensuring a robust and versatile platform. This design strategy also opens avenues for further innovation, as new modules can be developed to enhance system capabilities without overhauling the existing architecture.

Ultimately, adopting a modular approach in robotic platforms aligns perfectly with the goals of advancing research, fostering educational experiences, and enabling rapid prototyping. By leveraging the principles of modularity, developers can create adaptable and cost-effective platforms that meet the dynamic challenges of modern robotics.

\section{Notion of platforms}
Platforms can be considered a workspace that provides mechanical, electronic, software as well as energetic and power support for a project be it experimental or practical, but at the heart of this concept is "Abstraction", you don't need to understand how semiconductor or hardware works to work with an arduino for an instance, all you need is to be familiar with algorithms and programming; high level programming or even human language in our times of AI, no assembly or machine code anymore, but if you need to or want to go that low, it's still possible for you to do so, the arduino platform allows for all therefore:\\

\textbf{Definition.} A \emph{platform} is a deliberately stable core together with explicit contracts that enable independent parties to create, combine, and evolve extensions and applications over time without modifying the core.

\subsection{Platform vs. Product, Toolkit, Framework, Ecosystem}
% (Insert the comparison table here)

\subsection{Modular Robotic Platforms (MRP)}
An MRP exposes mechanical, electrical, and software contracts for safe composition of actuators, sensors, power, and behaviors.\\
"ASCE" platforms; what this thesis is about, follows an analogy: "if modularity is a universe, then our platforms allow freedom in both \textbf{space} and \textbf{time}"

\subsubsection{Space-Time Modularity}
Space = discipline (mechanical, electronics, software, logic/UX). 
Time = user maturity (consumer, learner, researcher, professional).

\subsection{The ASCE Platform Family}
% Formal tuple, shared philosophy, layers

\subsection{Minimum Viable Platform (MVPf)}
% 4 criteria list

\subsection{Axioms and Criteria}
% A1--A5 and testable criteria

\subsection{Platform Layers for Embedded Robotics}
% 8 layers list

\subsection{Contract Template}
% Per-interface template (IMU, ESC, Radio, etc.)

\subsection{Why ASCE is Different}
% Matrix argument and thesis claim



%ASCE Platforms is intedned to be a very modular platform that allows the users to customize the robots according to applications, this is by providing components at request.

%Such an approach would open the door to more potential ideas such as an interface to customize custom orders of such modular units to suite different professionl/industrial/developmental applications the user might require. it would also optimize the price of unit which may boost its commercial aspect.

%Initial ideas would only ivovle tires, custom prints on them or custom fins and sizes, even custom mecanuum wheels for different environments.






