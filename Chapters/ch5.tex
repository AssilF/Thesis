\chapter{Battery Power, Characteristics, and Longevity in Robotics}
\label{chap:battery}

\section{Introduction}
In robotic systems, batteries are not merely an energy source---they are a critical component that directly influences performance, autonomy, and reliability. Robotics applications, from mobile platforms to embedded sensor nodes, require power systems that are both energy-dense and robust against degradation. This chapter provides a comprehensive overview of battery power fundamentals, key battery parameters (such as capacity, C-rate, and internal resistance), energy conversion and loss mechanisms, charging methodologies, and advanced battery management techniques. Special emphasis is placed on how these concepts underpin robotic applications. Recent research highlights, including works by Roscher \emph{et al.} \cite{Roscher2011}, Rauf \emph{et al.} \cite{Rauf2022}, and Rahman and Alharbi \cite{Rahman2024}, are discussed throughout to present state-of-the-art insights.

\section{Fundamentals of Battery Power}
Batteries convert stored chemical energy into electrical energy, powering robotic operations through controlled electrochemical reactions. The instantaneous power output \(P\) of a battery is given by:
\begin{equation}
	P = V \times I,
\end{equation}
where \(V\) is the terminal voltage and \(I\) is the current. Over time, the total work \(W\) provided by the battery is:
\begin{equation}
	W = \int_{t_0}^{t_f} V(t) \cdot I(t) \, dt.
\end{equation}
These relationships are fundamental when designing power budgets for robotic missions, where continuous operation under variable load conditions is required.

\section{Key Battery Characteristics and Terminology}
A thorough understanding of battery performance involves multiple key parameters:

\subsection{Capacity and Energy Storage}
Battery capacity, measured in ampere-hours (Ah), represents the total electric charge a battery can store. The energy content \(E\) (in watt-hours, Wh) is given by:
\begin{equation}
	E = V \times Q,
\end{equation}
where \(Q\) denotes the capacity. In robotics, high capacity is crucial for extended missions, yet must be balanced against weight and size constraints.

\subsection{C-Rate and Operational Dynamics}
The C-rate is defined as the charge or discharge current normalized by the battery's nominal capacity. For example, a 1C rate implies that the battery discharges its full capacity in one hour. If a 2 Ah battery is discharged at 2 A, the rate is 1C; at 4 A, it is 2C. Higher C-rates can lead to rapid energy delivery, which is desirable for high-power maneuvers in robotics. However, they also induce thermal stress and accelerate degradation \cite{Rahman2024}.

\subsection{Internal Resistance and Voltage Drop}
The battery's internal resistance \(R_{\text{int}}\) is a key factor that affects its performance. Under load, the voltage drop across \(R_{\text{int}}\) can be expressed as:
\begin{equation}
	V_{\text{drop}} = I \times R_{\text{int}}.
\end{equation}
Consequently, the power loss due to internal resistance is:
\begin{equation}
	P_{\text{loss}} = I^2 \times R_{\text{int}}.
\end{equation}
These losses generate heat and reduce the effective energy delivered, making thermal management a priority in robotic power systems.

\section{Energy Conversion and Loss Mechanisms}
Understanding energy conversion in batteries is essential for efficient system design. While the chemical-to-electrical conversion is highly effective, several loss mechanisms reduce the net energy output:
\begin{itemize}
	\item \textbf{Resistive Losses:} As described by Equation (4), losses due to internal resistance diminish available power.
	\item \textbf{Thermal Losses:} Elevated operating temperatures increase resistive losses and can lead to accelerated degradation.
	\item \textbf{Electrochemical Inefficiencies:} Side reactions and incomplete charge transfer reduce the effective capacity.
\end{itemize}
In robotics, minimizing these losses directly translates to longer operational time and higher reliability.

\section{Battery Charging Techniques}
Efficient charging is critical for maintaining battery health and longevity in robotic systems.

\subsection{The CCCV Charging Protocol}
Most modern batteries, particularly lithium-ion types, are charged using the Constant Current-Constant Voltage (CCCV) method. Initially, a constant current is applied until the battery voltage reaches a set threshold. Thereafter, the charger maintains a constant voltage, and the current gradually decreases. This protocol minimizes overcharging and thermal stress, thereby reducing degradation over multiple cycles.

\subsection{Advanced Charging Strategies}
Beyond CCCV, advanced techniques such as pulse charging and adaptive charging algorithms are under investigation. These strategies aim to optimize charge acceptance and reduce aging by dynamically adjusting the charging profile based on battery temperature, state-of-charge (SOC), and internal resistance feedback \cite{Rauf2022}.

\subsection{Battery Management Systems (BMS)}
A sophisticated BMS is crucial for robotics, where the battery must be monitored and controlled in real time. Modern BMS architectures integrate sensor arrays to measure voltage, current, temperature, and impedance. By using data-driven methods and machine learning, the BMS can predict state-of-health (SOH) and remaining useful life (RUL), and adjust charging parameters to extend battery life \cite{Rahman2024}.

\section{Battery Degradation Mechanisms}
Battery degradation is an inevitable phenomenon that impacts the performance and safety of robotic systems. Key degradation factors include:

\subsection{Cycling Degradation}
\res{I may include some graphics and schematics on the mechanism of battery degradation here ?}

Frequent charge-discharge cycles lead to capacity fade. Repeated cycling results in the gradual breakdown of electrode materials and the formation of passivation layers, which increases internal resistance and reduces energy storage capacity \cite{Roscher2011}.

\subsection{Thermal Effects}
High operating temperatures accelerate degradation through enhanced chemical reactions. Excessive heat can lead to electrolyte breakdown and structural damage within the cell. Robotic systems operating in harsh environments must implement active or passive thermal management strategies to mitigate these effects.

\subsection{Depth of Discharge and Calendar Aging}
Deep discharges stress the battery and reduce its overall life span. Additionally, even without cycling, batteries undergo calendar aging due to slow chemical reactions over time. Both factors must be carefully managed in robotics to ensure consistent performance.

\subsection{Mathematical Modeling of Degradation}
Quantitative models for battery degradation are crucial for predicting SOH and RUL. A common expression for SOH is:
\begin{equation}
	\text{SOH} = \frac{Q_{\text{actual}}}{Q_{\text{nominal}}} \times 100\%,
\end{equation}
where \(Q_{\text{actual}}\) is the current capacity and \(Q_{\text{nominal}}\) is the initial capacity. More sophisticated models incorporate changes in internal resistance and other parameters, enabling predictive maintenance and adaptive energy management in robotics \cite{Roscher2011}.

\section{State-of-Health (SOH) and Remaining Useful Life (RUL) Estimation}
SOH and RUL are vital metrics for battery performance monitoring:
\begin{itemize}
	\item \textbf{State-of-Health (SOH):} Represents the current capacity relative to the nominal capacity. It is a direct measure of degradation and informs about the battery's ability to deliver power.
	\item \textbf{Remaining Useful Life (RUL):} Estimates the number of cycles or the time remaining before the battery degrades to a level that necessitates replacement.
\end{itemize}
Advanced techniques for SOH and RUL estimation leverage both physics-based models and data-driven approaches, including Kalman filters and machine learning algorithms. These methodologies have been successfully applied to optimize battery performance in robotics, where real-time estimation is essential \cite{Rauf2022,Rahman2024}.

\section{Implications for Robotics}
In robotics, battery performance directly affects operational efficiency, endurance, and reliability. Key considerations include:
\begin{itemize}
	\item \textbf{Energy Density vs. Weight:} High energy density is critical for extending mission duration, yet must be balanced against weight constraints in mobile robotic platforms.
	\item \textbf{Thermal Management:} Robust cooling strategies are essential to mitigate heat-induced degradation, especially in high-performance or outdoor robotics.
	\item \textbf{Modular Battery Architectures:} For robotics, designing batteries that are easily replaceable or upgradable enables flexibility and extended system lifetimes.
	\item \textbf{Adaptive Power Management:} Incorporating real-time SOH and RUL estimations allows robotic systems to adapt their energy usage and charging cycles dynamically, thereby maximizing operational efficiency.
\end{itemize}

\section{Case Studies and Applications in Robotics}
Several robotic systems have successfully integrated advanced battery management to enhance performance:
\begin{itemize}
	\item \textbf{Mobile Robotics:} Autonomous rovers and drones require batteries that can handle high C-rates for rapid acceleration and deceleration. The integration of BMS with predictive algorithms ensures that these platforms can operate reliably in dynamic environments.
	\item \textbf{Embedded Systems:} In sensor networks and IoT devices, battery longevity is paramount. Techniques such as adaptive charging and low-power management extend the operational life of these systems.
	\item \textbf{Educational and Research Platforms:} Open-source robotics platforms benefit from modular battery architectures, allowing students and researchers to experiment with different power management strategies and observe the effects of degradation in real time.
\end{itemize}
These examples underscore the importance of battery optimization in achieving the desired performance and reliability in diverse robotic applications.

\section{Future Research Directions and Challenges}
Despite significant advancements, several challenges remain:
\begin{itemize}
	\item \textbf{Improving Degradation Models:} There is a need for more accurate and robust models that can account for the complex interplay of factors affecting battery degradation in real-world robotics applications.
	\item \textbf{Integration of AI in BMS:} Further integration of machine learning and artificial intelligence could enable more precise real-time monitoring and adaptive control, leading to better SOH and RUL predictions.
	\item \textbf{Sustainable Battery Technologies:} Research into new battery chemistries and materials that offer higher energy density and longer life spans will be crucial for the future of robotics.
	\item \textbf{Miniaturization and Modularity:} Developing modular battery packs that can be easily integrated into various robotic platforms without compromising performance remains an ongoing engineering challenge.
\end{itemize}
Addressing these challenges will be key to developing the next generation of robotic systems with enhanced autonomy and reliability.

\section{Conclusion}
This chapter has provided an in-depth examination of battery power fundamentals, key characteristics, and longevity considerations within the context of robotics. We have discussed the electrical principles that govern battery operation, explored critical parameters such as capacity, C-rate, and internal resistance, and detailed the mechanisms of energy loss and degradation. Advanced charging techniques, sophisticated battery management systems, and modern degradation models have been reviewed, emphasizing their applications in robotics. By integrating theoretical insights with state-of-the-art research, this chapter lays the foundation for designing robust, efficient, and long-lasting power systems essential for next-generation robotic platforms.


\chapter{Discovery and pairing system:}
\label{chap:discovery}

We've already established that our notion of "generalized platforms" implies cross-platform functionlaity, we need us an "on-platform" system that can identify other pair-able platforms nearby/from the cloud, their types, fetch their characteristics and parameters and give the user the ability to exploit such features. How can we go on about such a system?

We could initially develop our own physical link, protocol and system, or we can use something cheap and robust and already well documented such as ESP-NOW, since most of our robots are ESP32 based or incorporate ESP32 for connectivity, this makes it the perfect choice. ESP-NOW should give us a ready to use physical link and protocol, and all is left for us to develop is the pairing system and identification system, which is mostly software after we make the boards on BULKY and THEGILL self identify the plugged in peripherals, and Microtomba and others being breadboard based and manually configured, we can implement such system on every platform but with different interfaces, where we have a UI we can control behaviours from the robot itself such the case of BULKY, or we can use a controller for such behaviours too such is the case with ILITE, we can make all intercommunicate to create synergy or emergeant behaviour or a hive system, let's call it ASCE OPENswarm system, this would raise certain concerns like safety of operation, how do we detect unsafe scenarios and raise an exception to the user ? how do we make certain features apply to only a certain combination of platforms ? so on so on.

%note on the design, architecture, diagram and philosophy of such system
%note on how we will implement it differently
%note on the software implementation
%note on the user experience and interface and safety
%add diagram behind philosophy
%add technical diagram
%refer to code
%add experiments
%add discussion


\chapter{Hexagonal Architecture for Robotic Systems}
\label{chap:hexagonal-robotics}

Modular robotic platforms benefit from a structural philosophy that isolates concerns while enabling reuse and evolution across
mechanical, electrical, and software domains. The hexagonal architecture—also called the ports-and-adapters style—provides
such a philosophy by treating each subsystem as an independent ``hexagon'' exposing well-defined ports to the outside world.
Cockburn describes ports as ``driving adapters that isolate the application'' \cite{cockburn2005hexagonal}; we use these
adapters to separate robot hardware from the behaviours that drive it. Parnas noted that ``each module hides its secrets from
the others'' \cite{parnas1972}, which in our context means a gearbox or board reveals only its advertised ports while masking
implementation details. Meyer adds that ``clients must ensure preconditions'' \cite{meyer1992}; by publishing those
preconditions alongside every port, we prevent misuse and enable independent evolution of modules. In this chapter we extend
the hexagonal concept beyond software into robot-building, composing mechanical drives, circuit boards, microcontrollers, and
peripheral modules through contractual ports. The goal is not only architectural elegance but a practical path toward
scalable, versionable platforms that can adapt to new behaviours and physical configurations without revisiting foundational
design decisions.

\section{Mechanical Hexagons}
Parnas's maxim that ``each module hides its secrets'' \cite{parnas1972} guides the physical layer of our platform. Gearboxes
equipped with speed encoders expose rotational-speed and load ports, enabling standardized feedback loops regardless of gear
ratio. Craig reminds us that ``feedback from encoders closes the loop'' \cite{Craig2005}; accordingly, every mechanical hexagon
exports encoder counts so that higher layers can stabilize motion. Siciliano observes that ``mounting interfaces standardize
kinematic chains'' \cite{Siciliano2009}, inspiring our contracts for bolt patterns and center-of-mass references. By
encapsulating torque limits, precision classes, and mounting matrices inside mechanical hexagons, upstream electronics or
control software can rely upon consistent mechanical behaviour. This contract-driven view lets platforms swap drive units or
chassis elements without redesigning downstream electronics or software, much as software modules can be replaced when their
interfaces remain stable.

\begin{figure}[H]
  \centering
  \fbox{\rule{0pt}{2in} \rule{3in}{0pt}}
  \caption{Exploded view of a gearbox hexagon with highlighted speed and torque ports.}
  \label{fig:mech_hex}
\end{figure}

\section{Circuit Hexagons}
Circuit boards act as the second layer of hexagons. The \emph{Bulker} board routes power to four motor channels and includes
quadrature inputs for encoders, but its current capability is limited. Conversely, the \emph{Giller} board provides higher
current outputs for four motors but omits encoder support. Each board exposes electrical ports that describe voltage limits,
connector pinouts, current envelopes, and available sensing lines. By treating boards as interchangeable hexagons, a developer
selects the appropriate circuit for a task while the rest of the system interacts only with the declared ports, enabling
versioning of electronics independently from mechanics or software. Meyer's insistence that ``clients must ensure
preconditions'' \cite{meyer1992} motivates the documentation of allowable voltage and current ranges for every port.

\begin{figure}[H]
  \centering
  \fbox{\rule{0pt}{2in} \rule{3in}{0pt}}
  \caption{Comparison of Bulker and Giller circuit hexagons showing differing encoder and current ports.}
  \label{fig:circuit_hex}
\end{figure}

\section{Microcontroller Hexagons}
Baldwin and Woodard describe platforms as having ``a stable core and variable periphery'' \cite{baldwin2009platforms}. In our
design, the ESP32 microcontroller forms part of that stable core. Bulker and Giller both employ the ESP32, which offers
Wi-Fi, Bluetooth, and ESP-NOW communication. We abstract the ESP32 as another hexagon whose ports specify compute resources,
communication protocols, and timing guarantees. Firmware adapters map these ports to the mechanical and circuit hexagons,
isolating application logic from device-specific details. By keeping communication ports stable, processor upgrades or protocol
extensions occur without cascading changes across the system, facilitating long-term maintenance.

\begin{figure}[H]
  \centering
  \fbox{\rule{0pt}{2in} \rule{3in}{0pt}}
  \caption{ESP32 hexagon illustrating wireless and timing ports connected to circuit and mechanical hexagons.}
  \label{fig:mcu_hex}
\end{figure}

\section{Peripheral Hexagons}
Gawer and Cusumano argue that platform leaders ``nurture complementary innovation'' \cite{gawer2002platformleadership}. Our
peripheral hexagons embody that idea: the \emph{Mech'iane} subsystem mounts to Giller and implements an articulated robotic arm
that contributes complementary capabilities. Its ports include motor commands, inverse-kinematics services, and safety
interlocks. Tiwana observes that ``architecture and governance jointly shape platform evolution'' \cite{tiwana2014platformecosystems};
by defining the ports of peripherals, we govern how external modules may extend the platform. Because Giller exposes a uniform
motor-control port, Mech'iane plugs into it without changing the rest of the platform. Additional peripherals—such as range
sensors or grippers—become new hexagons whose adapters translate their signals into the platform's contracts. Framing
high-level behaviours as hexagons thus allows complex capabilities to be added or replaced without re-engineering the core system.

\begin{figure}[H]
  \centering
  \fbox{\rule{0pt}{2in} \rule{3in}{0pt}}
  \caption{Mech'iane peripheral hexagon attached to Giller through standard motor-control ports.}
  \label{fig:peripheral_hex}
\end{figure}

\section{System Composition}
The overall robot emerges by composing hexagons: mechanical drives connect to circuit boards, which connect to the ESP32 hexagon, which in turn connects to peripherals and communication modules. Each connection occurs through a port whose contract specifies functionality, timing, and error behavior. Yim et~al. report that "standardized connectors permit dynamic reassembly" \cite{Yim2007}; our port contracts extend this principle beyond mechanics to electronics and software. The resulting architecture respects boundaries between mechanics, power, electronics, and software, yet provides explicit seams where behaviours and communication strategies can be swapped. The system therefore evolves through adaptation rather than overhaul.

\begin{figure}[H]
  \centering
  \fbox{\rule{0pt}{2in} \rule{3in}{0pt}}
  \caption{System diagram composing mechanical, circuit, microcontroller, and peripheral hexagons.}
  \label{fig:system_composition}
\end{figure}

\section{Discussion}
The project began with a challenge common to many robotics efforts: how to scale and version hardware and software without
entangling mechanical, power, and behavioural considerations. By framing each subsystem as a hexagon with contractual ports, we
resolved this challenge. Mechanical modules could be replaced, circuit boards upgraded for current capacity, communication
protocols switched, and peripherals added—all without cascading redesign. The hexagonal architecture thus operationalises the
research goal of achieving modular, evolvable platforms across mechanical, electronic, power, and communication domains,
demonstrating that the philosophy of ports and adapters extends naturally into robotics.

\section*{Research Question and Hypothesis}
\textbf{RQ1:} We surely need to develop a concept, or a philosophy that allows for scalability, versioning, and modularity of our platforms. This should extend to mechanical aspects, Software aspects, Electronics aspect, Power aspects, and communications and behavior aspects.\\
\textbf{H1:} Hexagonal architecture widely used in software and databases, gives us a solid philosophy for isolating system, or platforms, into their own entity, while exposing ports to handle external entities, via adapters. Modifying this concept to our project's constraints and objectives should give us our intended philosophy.

