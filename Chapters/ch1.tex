\chapter{Foundational Concepts in Robotics, Kinematics, and Interdisciplinary Integration}
\label{chap:foundations}
\res{This chapter isn't finalized, it's me trying to imagine an initial structure to be edited over time, I am not sure if I should provide superficial insights on robotics to put the reader of this thesis on the same page with me as I document my work more in following chapters, or if I should dive deeper into each part of each chapter which will result in a very rich yet a long thesis which may not even be read in the end}

This chapter establishes a rigorous theoretical framework for the work presented in this thesis. It delves into advanced kinematic models, surveys the multidisciplinary nature of robotics, and outlines the research methodology grounded in abstraction and modularity. The discussion is supported by key academic references and mathematical formulations, setting the stage for the detailed design and experimental work in later chapters.

\section{Overview of Robotics}
Robotics is an interdisciplinary field that has evolved from simple automation to the development of complex systems with autonomous decision-making capabilities. This evolution is driven by:
\begin{itemize}
    \item \textbf{Sensor and Actuator Advances:} Modern robots leverage cutting-edge sensors and actuators that provide high-resolution environmental data and precise control \cite{Craig2005,Siciliano2009}.
    \item \textbf{Computational Power and Algorithms:} Enhanced processing capabilities and advanced algorithms, including machine learning and adaptive control, allow robots to perform real-time analysis and complex tasks.
    \item \textbf{Modularity and Open-Source Development:} The trend toward modular design enables scalable, reconfigurable platforms that can be tailored for research, education, and even consumer applications. This paradigm shift facilitates collaborative development and rapid innovation.
\end{itemize}

Academic studies often address challenges such as system integration, sensor fusion, and the development of robust control algorithms. Foundational works by Siciliano \emph{et al.} \cite{Siciliano2009} and Craig \cite{Craig2005} have significantly influenced modern robotics by providing theoretical and practical insights that continue to shape the field.

\section{Fundamentals of Kinematics in Robotics}
Kinematics is central to understanding and controlling the motion of robotic systems. This section presents detailed models for both forward and inverse kinematics.

\subsection{Forward Kinematics}
Forward kinematics is the process of determining the position and orientation of a robot's end-effector from known joint parameters. The Denavit-Hartenberg (D-H) convention is a standard method used to express the transformation for each joint. For joint \( i \), the transformation matrix is:
\[
\mathbf{T}_i = \begin{bmatrix}
\cos\theta_i & -\sin\theta_i \cos\alpha_i & \sin\theta_i \sin\alpha_i & a_i \cos\theta_i \\
\sin\theta_i & \cos\theta_i \cos\alpha_i  & -\cos\theta_i \sin\alpha_i & a_i \sin\theta_i \\
0            & \sin\alpha_i              & \cos\alpha_i              & d_i \\
0            & 0                         & 0                         & 1
\end{bmatrix}
\]
The cumulative transformation to the end-effector is given by:
\[
\mathbf{T}_{\text{end-effector}} = \prod_{i=1}^{n} \mathbf{T}_i
\]
This formulation is vital for simulation and trajectory planning, and it is widely implemented in robotics software such as MATLAB and ROS \cite{Spong2006}.

\subsection{Inverse Kinematics}
Inverse kinematics seeks the joint parameters that achieve a desired end-effector pose. Unlike forward kinematics, inverse kinematics is generally nonlinear and may yield multiple or no solutions due to the presence of singularities. Two primary approaches include:
\begin{itemize}
    \item \textbf{Analytical Methods:} Closed-form solutions are feasible for simpler kinematic chains but quickly become intractable with increased degrees of freedom.
    \item \textbf{Numerical Methods:} Iterative algorithms such as the Newton-Raphson method or methods based on the Jacobian matrix are employed to converge on a solution.
\end{itemize}
The Jacobian matrix \( \mathbf{J} \) is defined as:
\[
\mathbf{J}(\mathbf{q}) = \frac{\partial \mathbf{x}}{\partial \mathbf{q}}
\]
where \( \mathbf{q} \) represents the joint variables and \( \mathbf{x} \) the end-effector position. It relates the joint velocities \( \dot{\mathbf{q}} \) to the end-effector velocity \( \dot{\mathbf{x}} \):
\[
\dot{\mathbf{x}} = \mathbf{J}(\mathbf{q}) \dot{\mathbf{q}}
\]
Advanced studies in redundancy resolution and singularity avoidance \cite{Lynch2017} further extend the practical application of these models.

\section{Interdisciplinary Nature of Robotics}
Robotic systems integrate diverse fields, each contributing unique methodologies and insights.

\subsection{Mechanical Engineering}
Mechanical engineering underpins the design and fabrication of the physical structure. Key considerations include:
\begin{itemize}
    \item \textbf{Structural and Material Analysis:} Techniques such as finite element analysis (FEA) assess the durability and performance of robot components.
    \item \textbf{Mechanism Design:} Analyzing kinematic chains, joint configurations, and mobility mechanisms is essential for ensuring reliable operation under various conditions.
\end{itemize}
Classical texts, such as those by Spong \emph{et al.} \cite{Spong2006}, provide deep insights into these principles.

\subsection{Electrical and Electronics Engineering}
This discipline focuses on:
\begin{itemize}
    \item \textbf{Circuit and Sensor Integration:} Designing custom PCBs, interfacing diverse sensors, and ensuring efficient power management are critical, especially for platforms requiring long battery life.
    \item \textbf{Signal Processing:} Filtering and interpreting sensor data to provide accurate feedback for control algorithms.
\end{itemize}
These topics are discussed extensively in foundational works like \cite{Craig2005}.

\subsection{Computer Science and Control Theory}
Software and control algorithms are the brain of robotic systems:
\begin{itemize}
    \item \textbf{Control Strategies:} The implementation of PID controllers, model predictive control (MPC), and adaptive algorithms enables stable and responsive behavior.
    \item \textbf{Software Architectures:} Developing modular, scalable, and real-time software platforms is essential for integrating various subsystems. Open-source frameworks like ROS have revolutionized this integration.
\end{itemize}
For a comprehensive treatment, refer to \cite{Siciliano2009}.

\subsection{Systems Engineering}
Systems engineering ensures that all components operate harmoniously:
\begin{itemize}
    \item \textbf{Integration and Testing:} Systematic approaches for validating interoperability and performance are vital.
    \item \textbf{Lifecycle Management:} Structured methodologies help manage the complexity of designing, developing, and deploying advanced robotic systems.
\end{itemize}
This holistic perspective is crucial for the successful implementation of modular and adaptable platforms.

\section{Research Approach}
The methodology of this thesis is built on the principles of abstraction and modularity. Rather than developing fixed-function systems, the focus is on creating versatile platforms with the following core components:
\begin{itemize}
    \item \textbf{Abstraction:} By decoupling hardware from software, the design allows for independent evolution and easy integration of new technologies.
    \item \textbf{Modularity:} Systems are conceived as collections of interchangeable modules, enabling rapid prototyping, iterative testing, and scalability. This approach is supported by recent research on modular self-reconfigurable robots \cite{Yim2007}.
    \item \textbf{Theory-Practice Integration:} Theoretical models, such as the kinematic formulations presented earlier, are used to inform design choices and are validated through extensive simulations and experiments.
\end{itemize}

This integrated research strategy ensures that the platforms developed are not only scientifically sound but also practical and adaptable to various applications. Future work may extend these concepts to explore self-reconfigurable systems and advanced adaptive control methods.

\section{Conclusion}
This chapter has established a superficial robotics foundation for the thesis. By exploring advanced kinematic models, the interdisciplinary framework that supports modern robotics, and a methodologically rigorous research approach, the groundwork has been laid for the subsequent chapters. The next sections will build on these concepts by detailing the design, implementation, and experimental validation of the modular robotic platforms, thereby contributing to the evolution of accessible and adaptable robotics.

