\chapter{Electronic Foundation}
\section{Role of Electronics in a Modular Robotic Platform}

In a Modular Robotic Platform (MRP), the electronic layer acts as the interface between mechanical structures and software logic. It is responsible for power distribution, signal acquisition, actuation, communication, and user interaction. In contrast to monolithic robotic systems, modular platforms must tolerate frequent reconfiguration, heterogeneous peripherals, and evolving performance requirements without requiring redesign of the core electronics.

The electronic architecture presented in this thesis emphasizes abstraction, reuse, and robustness. By relying on widely available components, standardized signaling protocols, and scalable microcontroller architectures, the platforms achieve electronic modularity across diverse robotic systems.

\section{Power Architecture and Energy Storage}

\subsection{Battery Technologies}

Mobile robotic platforms require compact, high-energy-density power sources. Two battery chemistries are employed:

\begin{itemize}
	\item \textbf{Lithium-Ion (Li-Ion):} Used where mechanical robustness, stable discharge, and long cycle life are required.
	\item \textbf{Lithium Polymer (LiPo):} Selected for applications requiring high discharge rates and flexible form factors, such as mobile and aerial platforms.
\end{itemize}

Both battery types are widely adopted in robotics and embedded systems due to their favorable energy-to-weight ratios \cite{horowitz2015}.

\subsection{Voltage Regulation}

Robotic systems integrate components operating at different voltage levels. Step-down (buck) converters are therefore used to regulate battery voltages to stable logic and peripheral supply rails. Popular and well-documented modules such as Mini360 and regulators based on the LM2596 family are employed due to their high efficiency, thermal stability, and widespread availability \cite{ti_lm2596}.

Distributed regulation improves fault tolerance and enables independent scaling of subsystems.

\section{Signal Integrity and Noise Management}

Robotic electronics often combine low-level analog signals with high-current switching devices. To ensure reliable operation, the following practices are applied:

\begin{itemize}
	\item Local decoupling capacitors near IC power pins
	\item Separation of high-current and low-signal paths
	\item Short signal traces for high-speed or sensitive signals
	\item Logical grounding strategies to minimize noise coupling
\end{itemize}

These techniques are well established in embedded and mixed-signal system design \cite{williams2004}.

\section{Input and Output Expansion}

\subsection{Digital Expansion}

Shift registers are used to expand digital outputs while minimizing GPIO usage. By serially shifting bit patterns, multiple outputs can be controlled using a small number of microcontroller pins. This approach is effective for driving LEDs, relays, and generating software-based PWM signals \cite{monk2017}.

\subsection{Analog Expansion}

Analog multiplexers (MUX) allow multiple analog sensors to share a single ADC channel. This technique is especially useful when integrating numerous sensors without increasing microcontroller complexity or pin count.

\section{Actuation and Signal Amplification}

Microcontroller GPIO pins are insufficient for directly driving motors, solenoids, or high-power loads. Therefore, signal amplification stages are introduced:

\begin{itemize}
	\item \textbf{Bipolar transistors} for low-power switching
	\item \textbf{MOSFETs} for high-current applications due to low on-resistance
	\item Push-pull and H-bridge configurations for bidirectional control
\end{itemize}

These circuits form the basis of motor drivers and power switching stages across all platforms \cite{horowitz2015}.

\section{Sensors and Peripheral Modules}

The platforms integrate a wide range of sensors and peripherals using standardized interfaces such as I\textsuperscript{2}C, SPI, and UART. Common peripherals include:

\begin{itemize}
	\item Inertial Measurement Units (IMUs) for orientation and motion estimation; MPU6040
	\item Distance and proximity sensors (ultrasonic, infrared)
	\item Audio indicators such as buzzers and LEDs for feedback and debugging
\end{itemize}

Standardized interfaces ensure compatibility, reusability, and ease of integration.

\section{Microcontroller Architectures and Selection}

Microcontroller choice significantly impacts system scalability and performance.

\subsection{8-bit Microcontrollers}

Platforms such as µTomba use 8-bit AVR microcontrollers (e.g., Arduino Nano) for educational purposes. These devices offer simplicity and accessibility but are constrained by limited RAM, flash memory, and processing capability \cite{monk2017}.

\subsection{32-bit Microcontrollers}

More advanced platforms employ 32-bit microcontrollers, including the ESP32 and ESP32-S3 families. These devices, based on the Xtensa architecture, provide:

\begin{itemize}
	\item 32-bit word size for faster computation
	\item Larger addressable memory spaces
	\item Integrated communication peripherals
\end{itemize}

ESP32-based controllers offer hundreds of kilobytes of SRAM and multiple megabytes of flash, enabling complex firmware, graphical interfaces, and communication stacks to coexist \cite{espressif2023}.

\section{Implementation Across Platforms}

The electronic design philosophy is applied consistently across all robotic systems:

\begin{itemize}
	\item \textbf{\mu Tomba and Dronegaze:} Breadboard-based circuits for rapid prototyping and education
	\item \textbf{Bulky:} Custom PCB (BULKER32) for robustness and signal integrity
	\item \textbf{Thegill:} GillerS3 (ESP32-S3) and GillerDUO (ESP32 + RP2040) for high-performance and distributed control
\end{itemize}

This progression from breadboard to custom PCB reflects the MVP-to-MRP maturation process.

\subsection{BULKER32 design}

\subsection{GillerDUO design}

\subsection{GillerS3 design}

\subsection{Breadboard based designs}
Since the primary reason this option exists is for pedagogic and teaching applications, it is helpful to include component kits and guides with the MRPs explaining the mechanism of breadboards and how to use them as well as circuit examples and templates as can be shown in the figures below. 

with these 3 circuits, we have multiple options to realize MVPs that serve as customizable MRPs 

\section{ILITE Controller: Electronic Design Rationale}

ILITE serves as a unifying control interface across the proposed Modular Robotic Platforms (MRPs). Its electronic design prioritizes robustness, scalability, and usability while maintaining compatibility with heterogeneous robotic configurations. The controller integrates human input devices, sensing, display, and processing within a compact and power-efficient architecture.

\subsection{Human Interface Components}

ILITE integrates multiple input modalities to support intuitive and reliable user interaction.

\subsubsection{Analog Joysticks and potentiometers}

Two analog joysticks are used to provide continuous two-dimensional control inputs. These devices typically operate within a 0–3.3\,V or 0–5\,V range and draw negligible current (on the order of microamperes), making them well-suited for low-power embedded systems. Analog joysticks interface directly with the microcontroller’s ADC, enabling high-resolution input sampling and smooth control behavior \cite{horowitz2015}.

\subsubsection{Rotary Encoders}

Incremental rotary encoders are employed for precise user input and navigation. Encoders offer advantages over potentiometers in terms of durability and repeatability, as they are not subject to mechanical wear of resistive elements. Their digital quadrature outputs allow accurate position and direction detection with minimal processing overhead \cite{monk2017}.

\subsubsection{Buttons with Hardware Debouncing}

Push buttons are interfaced using a Schmitt trigger–based conditioning circuit implemented with a NOT gate, resistors, capacitors, and diodes. This approach introduces hysteresis and RC filtering to suppress contact bounce and noise at the hardware level. Hardware debouncing reduces software complexity and ensures reliable state transitions, particularly in noisy environments or during fast user interaction \cite{williams2004}.

\subsection{Visual Feedback Interface}

An SH1106-based OLED display connected via the I\textsuperscript{2}C bus is used to provide graphical feedback. OLED technology was selected due to its low power consumption, high contrast ratio, and wide viewing angle. The SH1106 controller supports resolutions sufficient for menus, telemetry, and debugging information while consuming only a few tens of milliamperes during active operation \cite{sh1106datasheet}.

The use of I\textsuperscript{2}C minimizes pin usage and allows the display to coexist with other peripherals on the same bus, reinforcing the modular philosophy of ILITE.

\subsection{Microcontroller Architecture and Word Size}

ILITE is designed around 32-bit microcontroller architectures, specifically the ESP32 and ESP32-S3 families. These devices are based on the Xtensa LX6/LX7 architecture and provide significant advantages over traditional 8-bit microcontrollers such as AVR-based devices.

Compared to 8-bit architectures, 32-bit microcontrollers offer:
\begin{itemize}
	\item Wider word size, enabling faster arithmetic and control computations
	\item Larger addressable memory spaces
	\item Native support for advanced peripherals and communication protocols
\end{itemize}

ESP32-class devices typically provide hundreds of kilobytes of SRAM and several megabytes of external flash, allowing complex firmware, graphical interfaces, and communication stacks to coexist within a single system \cite{espressif2023}.

In contrast, 8-bit AVR microcontrollers, such as those used in Arduino Nano–based systems, offer limited RAM and flash memory, making them more suitable for educational platforms such as µTomba but less appropriate for feature-rich controllers like ILITE \cite{monk2017}.

\subsection{Power and Current Considerations}

The electronic components of ILITE are selected to operate within safe and efficient current limits. Input devices and displays draw relatively low currents, while the microcontroller represents the primary power consumer, particularly during wireless communication or intensive computation.

ESP32-based systems are capable of dynamic power scaling and low-power modes, allowing ILITE to balance performance and energy efficiency. Voltage regulation and decoupling strategies discussed earlier ensure stable operation even under varying load conditions.

\subsection{Design Justification}

The combination of robust input conditioning, efficient display technology, and a 32-bit microcontroller architecture enables ILITE to function as a scalable and reusable controller across multiple robotic platforms. By abstracting hardware complexity while retaining performance headroom, ILITE embodies the electronic principles of the Modular Robotic Platform concept.

\section{Chapter Summary}

This chapter presented the electronic foundations of the proposed modular robotic platforms. Through careful selection of power architectures, signaling techniques, microcontroller families, and implementation strategies, the electronic layer supports scalability, robustness, and reuse. These foundations enable the software abstractions and communication frameworks discussed in the following chapter.
