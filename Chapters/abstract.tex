\chapter*{Abstract}
Modular robotics has emerged as a promising approach to address the limitations of traditional monolithic robotic systems in terms of adaptability, scalability, and maintenance. By enabling robots to be constructed from interchangeable functional units, modular robotic platforms offer the potential for rapid reconfiguration, fault tolerance, and task-specific customization. However, many existing modular robotic systems suffer from limited hardware interoperability, tightly coupled software architectures, and complex reconfiguration procedures that hinder practical deployment and scalability.

This thesis presents the design and implementation of a modular robotics platform that emphasizes both hardware and software modularity. The proposed platform is composed of standardized robotic modules that integrate actuation, sensing, control, and power interfaces, allowing modules to be physically and logically interconnected in multiple configurations. A layered software architecture is introduced to support dynamic module identification, inter-module communication, and distributed control, enabling plug-and-play functionality and reducing system integration effort.

The effectiveness of the proposed modular robotics platform is validated through a series of experimental evaluations focusing on reconfiguration capability, system scalability, and operational reliability. Experiments demonstrate that the platform can be reconfigured with minimal manual intervention while maintaining stable communication and control across different module arrangements. Performance metrics such as reconfiguration time, communication latency, and task execution success rate are used to assess system behavior under varying configurations.

The main contributions of this thesis include: (i) the design of a unified hardware interface that supports modular robotic assembly, (ii) a flexible software framework enabling dynamic reconfiguration and distributed control, and (iii) experimental validation of a modular robotic platform demonstrating practical feasibility. The results indicate that the proposed approach provides a viable foundation for scalable and adaptable modular robotic systems.