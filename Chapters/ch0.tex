\chapter{Mechanical foundations}
\section{Role of Mechanical Design in Modular Robotic Platforms}

In a Modular Robotic Platform (MRP), mechanical design plays a fundamental role in enabling reusability, adaptability, and scalability. Unlike monolithic robotic systems, where mechanical components are tightly coupled to a single application, an MRP must support frequent reconfiguration while maintaining structural integrity and functional reliability.

The mechanical layer provides the physical substrate upon which electronic and software modules operate. Therefore, mechanical interfaces must be standardized, robust, and flexible enough to accommodate heterogeneous components such as sensors, actuators, power systems, and auxiliary structures. In the context of the ASCE platform family, mechanical modularity is treated as a first-class design constraint rather than a secondary optimization.

\section{Design Principles for Mechanical Modularity}

The mechanical foundations of the proposed platforms are guided by the following principles:

\subsection{Reusability}

Mechanical parts are designed to be reused across multiple robotic platforms whenever possible. Mounting patterns, enclosure dimensions, and structural elements follow repeatable geometries, allowing components to migrate between different MRPs without redesign.

\subsection{Standardized Interfaces}

Standard mechanical interfaces, such as mounting holes, brackets, and rails, enable consistent attachment of modules. This standardization reduces integration time and minimizes the risk of incompatibility during reconfiguration.

\subsection{Structural Safety and Load Distribution}

While modularity favors flexibility, mechanical safety remains critical. Load-bearing elements are designed to withstand worst-case operating conditions, and mass distribution is carefully considered to preserve stability and maneuverability across different configurations.

\subsection{Scalability}

Mechanical designs allow incremental expansion. Additional modules can be mounted without altering the core structure, ensuring that the platform can evolve alongside user requirements.

\section{Mechanical Interfaces and Mounting Strategy}

To support modular assembly, the platforms adopt a layered mechanical interface strategy:

Primary structural frame: Provides rigidity and defines the robot’s main geometry.

Secondary mounting zones: Dedicated areas for sensors, actuators, and peripherals.

Payload regions: Reserved volumes for experimental or user-defined modules.

Mounting strategies emphasize accessibility and reversibility, enabling users to assemble, disassemble, and modify configurations with minimal tools. This approach aligns with the MRP objective of rapid prototyping and iterative development.

\section{Platform-Specific Mechanical Implementations}

Although each platform targets different operational environments, all share the same modular mechanical philosophy.

\subsection{uTomba}

uTomba adopts a minimalistic mechanical structure designed to prioritize accessibility and learning. The use of a breadboard-based layout eliminates rigid enclosure constraints, allowing users to physically interact with components and explore alternative configurations. This mechanical openness supports rapid experimentation and educational use while still maintaining a coherent platform identity.

\subsection{Bulky}

Bulky is designed as a compact terrestrial rover with an emphasis on sensor and actuator expandability. Its chassis integrates multiple mounting points for ultrasonic arrays, cameras, manipulators, and auxiliary modules. The mechanical design balances compactness with flexibility, enabling Bulky to operate both as a standalone system and as a peripheral-driven slave platform controlled by external computation units.

\subsection{Thegill}

Thegill extends mechanical modularity into amphibious environments. Its structure is engineered to support both terrestrial locomotion and aquatic propulsion while maintaining compatibility with additional systems such as robotic arms. Special consideration is given to sealing, buoyancy, and power routing, ensuring that modular components can be integrated without compromising environmental resilience.

\subsection{Dronegaze}

Dronegaze is conceived as a modular aerial test platform rather than a conventional flight-ready drone. Its mechanical design prioritizes safety and accessibility, functioning as a test jig that allows controlled experimentation with propellers, frames, and control parameters. This setup enables rapid validation of aerodynamic and control hypotheses while preserving reusability of components.

\subsection{ILITE}

ILITE's mechanical design is meant to serve as a drop resistant ergonomic model, easy on the hands and satisfying controls and feedback, with buttons and encoders accessible and long lasting, and for the least; aesthetically sound with ability to create multiple colour and style variants.

\section{Mechanical Modularity and MRP Validation}

Across all platforms, mechanical modularity enables the creation of multiple valid configurations that satisfy the criteria of an MRP or MVP. By reusing structural elements, adapters, mounting standards, and design principles, the platforms demonstrate that mechanical abstraction can coexist with domain-specific optimization.

This approach validates the mechanical layer of the MRP concept by showing that diverse robotic systems; terrestrial, amphibious, aerial, and educational; can be derived from unique or shared mechanical foundations without sacrificing functionality or safety.

\section{Conclusion}

This chapter presented the mechanical foundations of the proposed modular robotic platforms. By outlining key design principles, standardized interfaces, and platform-specific implementations, it has been shown that mechanical modularity provides a robust basis for reconfigurable robotic systems. These foundations enable the seamless integration of electronic and software layers, which are explored in the following chapter.