\chapter{Defining Platforms}\label{chap\:defining-platforms}
\section{Platform: Working Definition}\label{sec\:platform-definition}
A \emph{platform} is a deliberately stable core together with explicit contracts that enable independent parties to create, combine, and evolve extensions and applications over time without modifying the core.

Formally, we define a platform as the 5-tuple
\begin{equation}
\mathcal{P} = \langle \mathcal{C},; \Gamma,; \mathcal{E},; \mathcal{T},; \mathcal{G} \rangle,
\end{equation}
where $\mathcal{C}$ is the core, $\Gamma$ are the contracts (interfaces and guarantees), $\mathcal{E}$ are extensions, $\mathcal{T}$ are the tools, and $\mathcal{G}$ is governance.\\

These notions are commonly used in software design and computer science but we can apply the same philosophies behind them in our field of interest; robotics. 

\section{Origin of context}

The notion of a \emph{stable core with explicit contracts} emerges from three converging streams. First, modular design and information hiding established that software should be decomposed so that each module encapsulates design decisions behind stable interfaces, allowing implementations to vary without ripple effects \cite{parnas1972}. Second, \emph{Design by Contract} made those interface expectations explicit as preconditions, postconditions, and invariants, turning ``interfaces'' into formal \emph{contracts} between components \cite{meyer1992}. 

As software systems turned into industry platforms, researchers described architectures with a \emph{core} and a \emph{periphery of complements}, where the core's stability and the clarity of its interfaces enable independent evolution of complements. This core-complement view connects technical modularity to economic ecosystems and governance, giving a unified lens for platform design \cite{baldwin2009platforms}. 

In practice, the software-architecture community operationalized the idea through patterns such as the Hexagonal (Ports \& Adapters) Architecture: keep application logic independent of technology details, expose \emph{ports} as stable contracts, and attach \emph{adapters} for specific UIs, devices, or protocols. This preserves the core while enabling rapid replacement or addition of peripherals \cite{cockburn2005hexagonal}. 

Management and IS scholarship then integrated architecture with strategy and governance: how a platform owner maintains core stability, versioning, and compatibility while orchestrating third-party innovation. This literature articulates how interface design, API policies, and deprecation rules shape ecosystem health, providing the governance scaffolding that complements the technical contracts \cite{gawer2002platformleadership,tiwana2014platformecosystems}. 

Taken together, these threads justify the dissertation's framing: a platform is a deliberately stable core with explicit, versioned contracts that empower independent parties to create and evolve extensions over time, without modifying that core or building on top of it. This "core" may be rigid and not as flexible outside intended use cases, but we can add that missing dimension by making it "Modular"; the core is merely a solid kernel that allows for a huge variety of stock or third party modules (or adapters in the sense of hexagonal architecture) to plug in, letting the users shape it as they please (Spacetime liberty!).

\subsection{Platform vs. Product, Toolkit, Framework, Ecosystem}\label{subsec:platform-comparison}
The magic behind \textbf{\textit{ASCE}} platforms is that aside from being platforms, they may also behave as products, toolkits frameworks, and an ecosystem (Since modularity and hexagonal architecture philosophies allow for cross platform and emergent behavior).
\begin{table}[h]
\centering
\caption{Platform vs. related constructs}
\renewcommand{\arraystretch}{1.15}
\begin{tabularx}{\textwidth}{lXXXX}
\toprule
\textbf{Construct} & \textbf{Core intent} & \textbf{Who extends it?} & \textbf{Interface stability} & \textbf{Typical deliverables} \\
\midrule
Product   & Solve a specific use case             & Vendor             & n/a           & Device/app with fixed features \\
Toolkit   & Provide parts to assemble             & End user           & Low--Medium   & Components, examples \\
Framework & Invert control; fill hooks            & Developer          & Medium--High  & Base code + callbacks \\
Ecosystem & Community of complements              & Many parties       & Mixed         & Marketplaces, standards \\
\textbf{Platform} & \textbf{Stable core + contracts for many products} & \textbf{First\&third parties} & \textbf{High (versioned)} & \textbf{Core, interfaces, SDKs, docs, governance} \\
\bottomrule
\end{tabularx}
\end{table}

\section{Modular Robotic Platforms (MRP)}\label{sec:mrp}
A \emph{Modular Robotic Platform} (MRP) exposes mechanical, electrical, and software contracts for safe composition of actuators, sensors, power, and behaviors so that a wide range of robots can be built, upgraded, and repurposed without modifying the core.

\subsection{Space-Time Modularity}\label{subsec:space-time}
Space is \emph{discipline} (mechanical, electronics, software, logic/UX). Time is \emph{user maturity} (consumer, learner, researcher, professional). A high-quality MRP provides variation points across space and growth paths across time.

\begin{table}[H]
\centering
\caption{ASCE Space--Time matrix of modularity}
\renewcommand{\arraystretch}{1.2}
\begin{tabularx}{\textwidth}{lXXXX}
\toprule
\textbf{Time $\downarrow$ / Space $\rightarrow$} & \textbf{Mechanical} & \textbf{Electronics} & \textbf{Software} & \textbf{Logic \& UX} \\
\midrule
Consumer & Prebuilt mounts; safe geometry & Pre-wired modules & App presets & One-click modes, arming rules \\
Learner & Parametric parts & Labeled harnesses; rails & Config files; examples & Guided labs, wizards \\
Researcher/Dev & Custom frames; CoG guidance & Swappable drivers; EMI budget & Ports/adapters; SDK & Tests, CI, telemetry \\
Professional & Optimized rigs; stiffness specs & Power tiers; fusing; connectors & RT guarantees; schedulers & Toolchains, HIL, certification hooks \\
\bottomrule
\end{tabularx}
\end{table}

\section{Minimum Viable Platform (MVPf)}\label{sec:mvpf}
A \emph{Minimum Viable Platform} requires: (1) a stable, versioned contract with at least one extension point per relevant layer; (2) two distinct realizations exercising the contracts; (3) tooling to build/test/document extensions; and (4) governance for compatibility and deprecation.

\begin{table}[h]
\centering
\caption{MVPf checklist (evidence at submission time)}
\renewcommand{\arraystretch}{1.2}
\begin{tabularx}{\textwidth}{lX}
\toprule
\textbf{Criterion} & \textbf{Evidence artifact} \\
\midrule
Stable contract(s) & Interface spec (versioned), timing/error semantics, unit tests \\
Two realizations & e.g., Drone + Line-follower reusing core; alternate IMU/ESC drivers \\
Tooling & Templates, CI, documentation, simulator/HIL scripts \\
Governance & SemVer policy, compatibility matrix, deprecation schedule \\
\bottomrule
\end{tabularx}
\end{table}

\section{ASCE Contracts (Selected)}\label{sec:contracts}
This section provides concise, testable contract specifications for three key ports. Each contract includes functional semantics, timing, error model, and compliance tests. Versions use semantic versioning (e.g., \texttt{imu.v2}).

\subsection{IMU Port (\texttt{imu.v2})}\label{subsec:imu-contract}
\begin{description}
\item[Identifier \& Version] \texttt{imu.v2}.
\item[Functional Semantics] Provide linear acceleration $(a\_x,a\_y,a\_z)$ in $\mathrm{m/s^2}$ and angular velocity $(\omega\_x,\omega\_y,\omega\_z)$ in $\mathrm{rad/s}$; right-handed body frame; timestamps in microseconds since boot.
\item[Sampling] Required supported rates: {100, 200, 400} Hz; default 200 Hz. Report actual rate via \texttt{sampleRateHz()}.
\item[Timing Contract] End-to-end latency (sensor read $\rightarrow$ consumer) $< 2$ ms (typ.), jitter $< 500$ $\mu$s (95th percentile) at 200 Hz.
\item[Calibration] Bias/scale persisted via Store port; soft-iron/hard-iron optional; expose \texttt{setCalibration()} and \texttt{getCalibration()}.
\item[Error Model] Health states: \texttt{OK}, \texttt{DEGRADED} (auto-retrying), \texttt{FAILED} (requires reinit); provide error codes and backoff policy.
\item[Power/EMI] 3.3V logic; I\textsuperscript{2}C or SPI; max I\textsuperscript{2}C clock 400 kHz; bus hang recovery via clock stretching up to 9 pulses.
\item[Security/Integrity] Optional CRC in SPI; sequence counters for drop detection in DMA pipelines.
\item[Compliance Tests] (i) rate conformance; (ii) unit/axis sign tests; (iii) jitter histogram under CPU load; (iv) power brownout recovery; (v) hot-swap detection (if bus supports it).
\end{description}

\subsection{ESC Port (\texttt{esc.v1})}\label{subsec:esc-contract}
\begin{description}
\item[Identifier \& Version] \texttt{esc.v1}.
\item[Functional Semantics] Command normalized throttle $u\in[0,1]$; monotonic mapping to PWM or DSHOT. Provide \texttt{arm()}, \texttt{disarm()}, \texttt{write(u)}, \texttt{stop()}.
\item[Timing Contract] Update at 400 Hz (min). Command-to-output latency $<1$ ms typ.; report \texttt{lastWriteTime()}.
\item[Safety] Disarmed default; loss-of-signal watchdog $<50$ ms; min/max clamps; ramp-rate limits configurable.
\item[Error Model] \texttt{OK}, \texttt{TIMEOUT}, \texttt{BUS\_ERROR}, \texttt{OVERCURRENT} (if telemetried). Failsafe maps to \texttt{stop()}.
\item[Electrical] 3.3V PWM or digital protocol (e.g., DSHOT600); shared ground; specify rise/fall constraints for LEDC or timer units.
\item[Compliance Tests] (i) monotonicity sweep; (ii) timing scope traces; (iii) watchdog trip; (iv) arming sequence; (v) ramp-rate bound check.
\end{description}

\subsection{Radio Port (\texttt{radio.v1})}\label{subsec:radio-contract}
\begin{description}
\item[Identifier \& Version] \texttt{radio.v1}.
\item[Functional Semantics] Bidirectional datagrams with topics (e.g., \texttt{rc/*}, \texttt{telemetry/*}). Provide \texttt{publish(topic, payload)} and \texttt{subscribe(topic, handler)}.
\item[Timing Contract] Control path latency $<20$ ms one-way (95th percentile). Provide \texttt{rssi()} and link \texttt{quality()}.
\item[Error Model] Retries with exponential backoff; duplicate filtering via sequence numbers. Health: \texttt{OK}, \texttt{DEGRADED}, \texttt{DISCONNECTED}.
\item[Security] Optional pairing with nonce; packet authentication (HMAC) optional but recommended for professional tier.
\item[Compliance Tests] (i) range vs. latency curves; (ii) packet loss under interference; (iii) reconnect times; (iv) topic throughput fairness.
\end{description}

\section{Proof by Substitution: Swapping IMUs in Drongaz}\label{sec:proof-substitution-imu}
This section demonstrates that ASCE's ports \& adapters architecture allows replacing one IMU with another without changing the core (controllers, estimator, mixer, or telemetry schema).

\subsection{Port Interface (stable)}
\noindent\textbf{C++ Interface (header):}
\begin{lstlisting}[language=C++]
// Port: Imu (imu.v2)
struct ImuSample {
uint64\_t t\_us; // timestamp (us since boot)
float ax, ay, az;   // m/s^2
float gx, gy, gz;   // rad/s
};

enum class Health { OK, DEGRADED, FAILED };

class Imu {
public:
virtual \~Imu() {}
virtual bool begin() = 0;
virtual bool read(ImuSample& out) = 0; // non-blocking; returns false if no new sample
virtual float sampleRateHz() const = 0;
virtual Health health() const = 0;
};
\end{lstlisting}

\subsection{Two Adapters (variable)}
\noindent\textbf{Adapter A:} \texttt{Mpu6050Imu} implements \texttt{Imu} over I\textsuperscript{2}C.\\
\textbf{Adapter B:} \texttt{Icm20948Imu} implements \texttt{Imu} over SPI.

\begin{lstlisting}[language=C++]
class Mpu6050Imu : public Imu { /\* ... */ };
class Icm20948Imu : public Imu { /* ... \*/ };
\end{lstlisting}

\subsection{Dependency Injection into the Core}
The estimator depends only on the \texttt{Imu} port.
\begin{lstlisting}[language=C++]
class Estimator {
Imu& imu;
public:
explicit Estimator(Imu& ref) : imu(ref) {}
void tick() {
ImuSample s; if (imu.read(s)) { /\* predict-update using s \*/ }
}
};
\end{lstlisting}

\subsection{Build-Time Swap (no core edits)}
\begin{lstlisting}[language=C++]
// main.cpp (composition root)
\#ifdef USE\_ICM20948
static Icm20948Imu imu;
\#else
static Mpu6050Imu imu;
\#endif
static Estimator estimator(imu);
\end{lstlisting}

\subsection{Run-Time Swap via Factory (still no core edits)}
\begin{lstlisting}[language=C++]
// config.json (deployed via Store port)
// { "imu": { "driver": "icm20948", "rate\_hz": 400 } }

std::unique\_ptr<Imu> makeImu(const Config& cfg) {
if (cfg.imu.driver == "icm20948") return std::make\_unique<Icm20948Imu>(cfg);
if (cfg.imu.driver == "mpu6050")  return std::make\_unique<Mpu6050Imu>(cfg);
return nullptr; // TODO: handle error
}
\end{lstlisting}

\subsection{Invariants Preserved}
\begin{itemize}
\item Telemetry schema (units, axes, timestamps) unchanged (consumer code unaffected).
\item Controllers/mixer unaffected; only \texttt{Imu} adapter swapped.
\item RTOS task graph unchanged; timing contract verified against \texttt{imu.v2} tests.
\item No edits to core files (e.g., \texttt{FlightCore}, \texttt{Controllers}, \texttt{Mixer}, \texttt{Telemetry}).
\end{itemize}

\section{Contract Template (for New Ports)}\label{sec:contract-template}
Use the following template when specifying additional ports (e.g., \texttt{clock.v1}, \texttt{telemetry.v1}, \texttt{store.v1}).

\begin{enumerate}
\item \textbf{Identifier \& Version}: \texttt{<name>.v<major>}.
\item \textbf{Functional Semantics}: Inputs/outputs, coordinate frames, units, lifecycle.
\item \textbf{Timing Contract}: Rates, latency/jitter bounds, concurrency model, buffering.
\item \textbf{Error Model}: Health states, error categories, retry/backoff, degradation behavior.
\item \textbf{Safety}: Failsafe states, watchdogs, arming/disarming, clamping.
\item \textbf{Electrical/Mechanical (if applicable)}: Voltage/current, connectors, mounting patterns.
\item \textbf{Security/Integrity}: Auth, sequence numbers, CRCs, tamper evidence.
\item \textbf{Compliance Tests}: Automated checks, HIL procedures, acceptance thresholds.
\item \textbf{Deprecation Policy}: Compatibility window and migration notes.
\end{enumerate}

